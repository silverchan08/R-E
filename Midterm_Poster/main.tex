R&E 과제명 (영문)
조은찬∗1, Jeongwon Shin†1, 서보연‡1, 최민호§1, 김훈¶2, 진교택‖3 and 이재원∗∗4
1Researcher, Korea Scinece Academy of KAIST
2Supervisor, Department of Mechanical Engineering, LATEX University
3Co-Supervisor, Department of Computer Science, LATEX University
4Assistant, Department of Computer Science, LATEX University
Abstract. Each chapter should be preceded by an abstract (10–15 lines long) that summarizes the
content. The abstract will appear and be available with unrestricted access. This allows unregistered
users to read the abstract as a teaser for the complete chapter. As a general rule the abstracts will not
appear in the printed version of your book unless it is the style of your particular book or that of the
series to which your book belongs.
1 Introduction
In knot theory, arc index is a knot invariant defined by the minimal number of half planes in the arc presentation.
We find the arc index for the prime theta and handcuff curve(대충 prime 정의에논문 출처 박아주고) with up to
seven crossings. 뭐시기저시기
2 Theoretical Background
The link with n-components is an embedding of the disjoint union of n circles S1 ∪ · · · ∪ S1 in R3. 1-component link
is called a knot. The θ-curve 아귀찮아 이건 나중에
3 Research Methods and Procedure
Definition 1. In a handcuff curve, the vertex edge is an edge that is connected to both vertices.
Definition 2. In a handcuff curve, the link component is a union of the loops from each vertex to itself.
Theorem 1 (뭐시기뭐시기 출처모름). If L is an alternating and non-split link, then
α(L) = c(L) + 2.
∗email
†io25jellyfish@gmail.com
‡이메일
§이메일
¶이메일
‖이메일
∗∗이메일
1
Figure 1: In a handcuff curve, the process of pulling out the inner bouquet without affecting the rest.
Theorem 2 (뭐시기뭐시기 출처모름). For any spatial graph H,
α(H) ≤ c(H) + e + b,
where e is the number of the edge and b is the number of the bouquet.
Corollary. If H is a handcuff curve,
α(H) ≤ c(H) + 5.
Especially, if the link component of H is non-split,
α(H) ≤ c(H) + 3.
Proof. If H is a handcuff curve, the number of the edge is 3, and the number of the bouquet is at most 2. Thus the
first inequality holds since e = 3, b ≤ 2. If there is a bouquet, one of the loops can be pulled out without affecting
the rest. (See figure 1.) Then, if we remove the vertex edge of H, the remaining link component is a split link. Thus
there are no bouquet if the link component of H is non-split, and we have second inequality since e = 3, b = 0.
Proposition 1. For a handcuff curve H, let L be a link component of H. Then,
α(H) ≥ α(L) + 1.
Proof. In the arc presentation of H, let v1, v2 be the vertices of H. Then, there are half-planes that contain the
vertex edge of H. If we remove them, the remainder is the arc presentation of L, the link component of H. Since
the number of half-planes that contain the vertex edge is at least 1, we obtain
α(H) ≥ α(L) + (the number of half plane that contain vertex edge) ≥ α(L) + 1.
Using the Theorem 1, we obtain the following corollary.
Corollary. In the handcuff curve H, if its link component L is alternating and non-split, then
α(H) ≥ c(L) + 3.
2
Proof. Since L is alternating and non-split link, α(L) = c(L) + 2 by Theorem 1. Thus,
α(H) ≥ α(L) + 1 = (c(L) + 2) + 1 = c(L) + 3
according to Proposition 1.
Combining the above corollary and the corollary of Theorem 2, we have the following theorem.
Theorem 3. For the handcuff curve H, if the link component L is alternating and non-split, then
α(H) = c(L) + 3.
We now consider the Yamada polynomial(주석으로 야마다 달아주고) of the handcuff curve and investigate the
relationship between the difference of its maximum and minimum degrees and the arc index.
Definition 3. For a graph G = (V, E), where V is vertex set of G and E is edge set of G, let us define 2-variable
Laurent polynomial
h(G)(x, y) = X
F ⊂E
(−x)−|F |xμ(G−F )yβ(G−F )
where μ(G) and β(G) is the number of connected components of G and the first Betti number of G.
Then, the Yamada polynomial of a graph G, R(G) is defined by
R(G)(x) = h(G)(−1, −x − 2 − x−1).
It is known that, for some integer n, the product (−x)nR(G) (where R(G) is the Yamada polynomial of the
spatial graph G) is an ambient isotopy invariant. Moreover, the Yamada polynomial satisfies the following properties.
Theorem 4. For the Yamada polynomial, the following properties hold.
1. R(·) = −1
2. Let e be a non-loop edge of a graph G. Then, R(G) = R(G/e) + R(G − e), where G/e, G − e denote the graphs
obtained by contracting and deleting the edge e, respectively.
3. Let e be a loop edge of a graph G. Then, R(G) = −(x + 1 + x−1)R(G − e).
4. Let G1 ∪ G2 be a disjoint union of graphs G1 and G2. Then, R(G1 ∪ G2) = R(G1)R(G2).
5. Let G1 · G2 be a union of graphs G1 and G2 having one common point. Then, R(G1 · G2) = −R(G1)R(G2).
6. If G has an isthmus, then R(G) = 0.
Theorem 5. For the Yamada polynomial, the following properties hold.
1. 그림 그려서 넣어야함 각 crossing 부근을 뭐시기뭐시기 한거 그림.
Now, using the stacked tangle representation and the Yamada polynomial, we will prove the following theorem.
The following theorem gives a lower bound for the arc index in terms of the Yamada polynomial.
Theorem 6. Let ST be the closure of stacked tangle of theta curve or handcuff curve. Then,
spr(R(ST )) ≤ 2n + 2
where n is the number of crossings in ST and spr(f ) denotes the spread of f , the difference between the maximal
and minimal degrees of f .
3
Corollary. If G is a theta curve or handcuff curve, then
α(G) ≥ 5 + p4spr(R(G)) − 15
2 ,
except when G is the trivial theta curve.
Proof. Let ST be the closure of stacked tangle of theta curve or handcuff curve. Let cs, css, cn, n be the number
of simple cap, semi-simple cap, non-simple cap and the number of crossings in ST , respectively. We will prove the
theorem using the mathematical induction on the pair (cs + css, n), ordered lexicographically.
1. Basis cases
First, suppose cs + css = 0. Then, since there are no simple disks, ST must be the trivial theta curve. In this
case, the number of crossing is at least 1. Since the spread of Yamada polynomial of trivial theta curve is 4,
spr(R(ST )) = 4 ≤ 2n + 2.
Second, suppose n = 0. Then, ST must be the disjoint union of trivial handcuff graph and possibly some
circles(unlink). Since the Yamada polynomial of trivial handcuff is zero, spr(R(ST )) = 0 ≤ 2 = 2n + 2.
Hence, basis step is proven.
2. Inductive step
Assume that the theorem holds for all pairs (c′
s + c′
ss, n′) < (cs + css, n), and suppose cs + css > 0. Then the
top disk must have a semi-simple cap. Let the disk connected to the top disk be denoted by Ds. (Note that
if the top disk had only a non-simple cap, then there would be no simple or semi-simple cap at all.) Since Ds
is not top disk, there exists a disk D directly above Ds. Now we consider two cases, depending on whether
there exists a cap between Ds and D. First, suppose that there is no cap between Ds and D. Then, there are
two possibilities: there is an intersection between Ds and D, or there is not.
If there is no intersection between Ds and D, we can swap the position of Ds and D without affecting the
rest of the diagram. If there is an intersection between Ds and D, we can use the relation
R
 
− R
 
= (x − x−1)

R
 
− R
 
.
In ST , let S′
T , S0
T and S∞
T be the diagrams obtained by replacing the crossing with , and
, respectively. Since both R S0
T
 and R (S∞
T ) have cs simple disks, css semi-simple disks, and n − 1
crossings, their spread is at most 2(n − 1) + 2 = 2n by the induction hypothesis. Therefore, the spread of
(x − x−1) R S0
T
 − R (S∞
T ) is at most 2n + 2. It is thus sufficient to prove that the spread of R (S′
T ) is at
most 2n + 2. Now, observe that S′
T can be interpreted as the diagram obtained by swapping the positions
of Ds and D in ST . Therefore, in both cases, it is enough to prove that the inequality holds after this swap-
ping. However, after each swap, the distance between the top disk and Ds decreases. Hence, we can continue
swapping Ds with the disk directly above it until either the disk above Ds is the top disk, or there is a cap
between Ds and the disk above it. Then it suffices to consider the case where there is a cap between Ds and
D.
Now, consider the case where there is a cap between Ds and D. There are two cases: either Ds is simple or
it is non-simple. If Ds is simple, we can reduce the number of caps using Reidemeister moves, as illustrated
in Figure 2. Its spread is at most 2n + 2 by the induction hypothesis. Therefore, the spread of ST is at most
2n + 2. If Ds is non-simple, the number of caps can be similarly reduced using the Reidemeister moves, as
shown in Figure 3.
4
Hence, by the base cases and the inductive step, the theorem follows by mathematical induction.
Corollary follows by bounding the number of crossings n in terms of the arc index α(G). Suppose G is a theta
curve, let RST be a reduced stacked tangle with no nested caps. Since the stacked tangle has 2 non-simple disks
and α(G) − 3 simple disks, the following holds.
1. The number of crossings between simple disks is at most α(G)−3
2
.
2. The number of crossings between simple and non-simple disks is at most 2 · (α(G) − 3).
3. The number of crossings between the two non-simple disks is at most 2, except for the trivial theta curve.
However, for each cap, the two arcs that it connects have no crossing since RST has no nested caps. Moreover, any
pair of simple arcs is connected by at most one cap. Note that between the two non-simple disks, there are at most
two caps, except in the trivial theta curve case. Therefore, we obtain the following upper bound on n in terms of
α(G):
n ≤
α(G) − 3
2

+ 2(α(G) − 3) + 2 − (α(G) − 2) = 1
2 (α(G)2 − 5α(G) + 8).
Using the above theorem, we have the following inequality:
α(G) ≥ 5 + p4spr(R(G)) − 15
2 .
Now suppose G is a handcuff curve. Similarly, the following holds.
1. The number of crossings between simple disks is at most α(G)−3
2
.
2. The number of crossings between simple and non-simple disks is at most 2 · (α(G) − 3).
3. The number of crossings between the two non-simple disks is at most 1.
Note that, in contrast to the theta case, some two arcs in a handcuff curve can be connected to two caps simulta-
neously when a simple arc intersects a non-simple one, because there are two loops in a handcuff curve. Moreover,
since a handcuff curve has only one edge connecting the two vertices, there can be at most one cap between the
two non-simple disks. Thus, the number of crossings satisfies the same upper bound:
n ≤
α(G) − 3
2

+ 2(α(G) − 3) + 1 − (α(G) − 1 − 2) = 1
2 (α(G)2 − 5α(G) + 8),
and we obtain the following inequality:
α(G) ≥ 5 + p4spr(R(G)) − 15
2 .
Figure 2, 3에 각각 푸는 모습 보여줘야함. 아직 더 필요
Theorem 7. 뭐시기 출처 모름
4 Research Result
뭔가 나오긴 하겠지...............?
5
5 Conclusion
요약, 더 나아가서 어떻게 써먹을 수 있을지?
References
[1] Leslie Lamport. LATEX: A Document Preparation System. Addison Wesley, Reading, Massachusetts, second
edition, 1994.
[2] Donald E. Knuth. The TEXbook. Addison Wesley, Reading, Massachusetts, 1984.
6