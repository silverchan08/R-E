\documentclass{article}
\usepackage{graphicx} % Required for inserting images
\usepackage{kotex}
\usepackage{amsmath}

\title{R\&E 참고용}
\author{조은찬 24-097}
\date{June 2025}

\begin{document}

\maketitle

\section{Introduction}
$\mathbf{Definition}$ The grid diagram of a knot is a projection that consists only of horizontal and vertical lines, and the vertical line always passes above the horizontal line. (However, in this case, there must be exactly two points (excluding the vertices) that are vertically bent for each horizontal line (row) and vertical line (column), and the two vertices must be on the horizontal line.) \\ \\
$\mathbf{Definition}$ The THC-Cromwell matrix is the matrix that satisfies the following conditions.
\begin{enumerate}
    \item It is a $n\times(n+1)$ matrix with entries 0 and 1
    \item It has only two '1's in every row and column, except for two rows. These two rows have three '1's.
\end{enumerate}
$\mathbf{Theorem}$ Every Theta knot and Hand-Cuff knot has its corresponding THC-Cromwell matrix. \\ \\
$\mathbf{Definition}$ Let any exception row(with three '1's) $i$ and it's two outer '1's $j$, $k$. The Hun's Matrix of THC-matrix is $(n-1)\times(n-1)$ matrix which deleted row $i$ and column $j$,$k$.\\
$\mathbf{Theorem}$ 
Hun's matrix good good verygood
ASDF

\section{Proof in the case of theta curve}
Erase the vertices(dots) of the row which has 3 vertices and also erase connected vertices of the left and right erased vertices in the grid diagram. Then, we can make the matrix of this grid diagram by (section n).\\ If end vertices is erased when we delete the row, then by what vertices is erased in the other three vertices row in the grid diagram can make different matrix.
\begin{enumerate}
    \item \textbf{0}\\
    T-shaped figure is given.
    \item \textbf{1 (middle vertex)}\\
    Line-shaped figure is given.
    \item \textbf{1 (end vertex)}\\
    Line-shaped figure is given.
    \item \textbf{2 (two end vertices)}\\
    T-shaped figure is given.
    \item \textbf{2 (middle and end vertices)}\\
    Line-shaped figure is given.
\end{enumerate}

\begin{enumerate}
    \item \textbf{Line-shaped}\\
    Let the matrix is given.
    $$\begin{pmatrix}
        1 & 1 & 1 & 0\\
        0 & 1 & 1 & 1\\
        1 & 0 & 0 & 1
    \end{pmatrix}$$
    If you work on deleting rows and columns here, you can get the line-shaped figure of the following figure. If you convert this into a matrix according to (section n),
    $$\begin{pmatrix}
        1 & 1 \\
        0 & 1
    \end{pmatrix}$$
    is given.
    If we appropriately perform adding multiple of one row to another row and replacing the second row with the result, the determinant changes only by $\pm 1$, and the resulting matrix will be identity matrix. If we appropriately transform other theta curves, we can get the same result.
    \item \textbf{T-shaped}\\
    Let the matrix is given.
    $$\begin{pmatrix}
        1 & 1 & 1\\
        1 & 1 & 1\\
    \end{pmatrix}$$
    If you work on deleting rows and columns here, you can get the T-shaped figure of the following figure. If you convert this into a matrix according to (section n),
    $$\begin{pmatrix}
        1 & 1 & 1\\
        0 & 1 & 0\\
        0 & 0 & 1
    \end{pmatrix}$$
    If we appropriately perform adding multiple of one row to another row and replacing the second row with the result, the determinant changes only by $\pm 1$, and the resulting matrix will be identity matrix. If we appropriately transform other theta curves, we can get the same result.
\end{enumerate}
Therefore, $\pm 1$ will be the determinant of the theta curve's Hun's matrix.


\end{document}