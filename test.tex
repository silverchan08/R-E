\documentclass{article}
\usepackage{graphicx} % Required for inserting images
\usepackage{kotex}
\usepackage{amsmath}

\title{R\&E 참고용}
\author{조은찬 24-097}
\date{June 2025}

\begin{document}

\maketitle

\section{Introduction}
$\mathbf{Definition}$ The grid diagram of a knot is a projection that consists only of horizontal and vertical lines, and the vertical line always passes above the horizontal line. (However, in this case, there must be exactly two points (excluding the vertices) that are vertically bent for each horizontal line (row) and vertical line (column), and the two vertices must be on the horizontal line.) \\ \\
$\mathbf{Definition}$ The THC-Cromwell matrix is the matrix that satisfies the following conditions.
\begin{enumerate}
    \item It is a $n\times(n+1)$ matrix with entries 0 and 1
    \item It has only two '1's in every row and column, except for two rows. These two rows have three '1's.
\end{enumerate}
$\mathbf{Theorem}$ Every Theta knot and Hand-Cuff knot has its corresponding THC-Cromwell matrix. \\ \\
$\mathbf{Definition}$ Let any exception row(with three '1's) $i$ and it's two outer '1's $j$, $k$. The Hun's Matrix of THC-matrix is $(n-1)\times(n-1)$ matrix which deleted row $i$ and column $j$,$k$.\\
$\mathbf{Theorem}$ 
Hun's matrix good good verygood
ASDF

\section{Proof in the case of theta curve}
Erase the vertices(dots) of the row which has 3 vertices and also erase connected vertices of the left and right erased vertices in the grid diagram. Then, we can make the matrix of this grid diagram by (section n).\\
한 row를 지울 때 양옆의 점이 지워지면, 그에 따라 다른 세 개의 점을 가진 row의 무슨 점이 지워지는지에 따라 지움(용어가 있으면 좋겠다)을 시행한 grid diagram은 각자 다른 matrix을 낳을 수 있다.
\begin{enumerate}
    \item \textbf{0개}\\
    T자 모양의 도형이 나온다.
    \item \textbf{1개 (가운데 점)}\\
    선 모양의 도형이 나온다.
    \item \textbf{1개 (끝 점)}\\
    선 모양의 도형이 나온다.
    \item \textbf{2개 (끝 점 2개}\\
    T 모양의 도형이 나온다.
    \item \textbf{2개 (가운데와 끝 점)}\\
    선 모양의 도형이 나온다.
\end{enumerate}

\begin{enumerate}
    \item \textbf{선 모양}\\
    다음과 같은 matrix를 생각하자.
    $$\begin{pmatrix}
        1 & 1 & 1 & 0\\
        0 & 1 & 1 & 1\\
        1 & 0 & 0 & 1
    \end{pmatrix}$$
    여기서 Hun's matrix를 만드는 작업을 하면 다음과 같은 그림의 선을 얻을 수 있다.
    이를 (section n)에 따라 matrix로 변환하면,
    $$\begin{pmatrix}
        1 & 1 \\
        0 & 1
    \end{pmatrix}$$
    If we appropriately perform adding multiple of one row to another row and replacing the second row with the result, the determinant changes only by $\pm 1$, and the resulting matrix will be identity matrix. 다른 모양들도 적절히 변형시키면 같은 결과를 얻을 수 있다.
    \item \textbf{T 모양}\\
        다음과 같은 matrix를 생각하자.
    $$\begin{pmatrix}
        1 & 1 & 1\\
        1 & 1 & 1\\
    \end{pmatrix}$$
    여기서 Hun's matrix를 만드는 작업을 하면 다음과 같은 그림의 T 모양을 얻을 수 있다. 이를 (section n)에 따라 matrix로 변환하면,
    $$\begin{pmatrix}
        1 & 1 & 1\\
        0 & 1 & 0\\
        0 & 0 & 1
    \end{pmatrix}$$
    If we appropriately perform adding multiple of one row to another row and replacing the second row with the result, the determinant changes only by $\pm 1$, and the resulting matrix will be identity matrix. 다른 모양들도 적절히 변형시키면 같은 결과를 얻을 수 있다.
\end{enumerate}
Therefore, $\pm 1$ will be the determinant of the theta curve's Hun's matrix.


\end{document}