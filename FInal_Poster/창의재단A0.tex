% !TEX program = lualatex 
% Unofficial University of Cambridge Poster Template
% https://github.com/andiac/gemini-cam
% a fork of https://github.com/anishathalye/gemini
% also refer to https://github.com/k4rtik/uchicago-poster

\documentclass[final]{beamer}

% ====================
% Packages
% ====================

\usepackage[T1]{fontenc}
\usepackage{lmodern}
\usepackage[orientation=portrait,size=a0]{beamerposter}
\usetheme{gemini}
\usecolortheme{nott}
\usepackage{graphicx}
\usepackage{booktabs}
\usepackage{tikz}
\usepackage{pgfplots}
\pgfplotsset{compat=1.14}
\usepackage{anyfontsize}
\usepackage{kotex}
\usepackage{amsmath}
\usepackage{mathtools}
\usepackage{amssymb}
\usepackage[labelformat=empty]{caption}
\usepackage{subcaption}
\usepackage{algpseudocode}
\setbeamerfont{block title}{size=\Large}
\setbeamerfont{block body}{size=\Large}



\captionsetup[subfigure]{labelformat=empty}

% ====================
% Lengths
% ====================

% If you have N columns, choose \sepwidth and \colwidth such that
% (N+1)*\sepwidth + N*\colwidth = \paperwidth
\newlength{\sepwidth}
\newlength{\colwidth}
% \setlength{\sepwidth}{0.025\paperwidth}
% \setlength{\colwidth}{0.45\paperwidth}

% \newcommand{\separatorcolumn}{\begin{column}{\sepwidth}\end{column}}


% ====================
% Title
% ====================

%\title{Arc Indices of Theta-Curves and Handcuff Graphs}
%\titlegraphic[width=3cm]{figure/logo.png}

\title{The Determinant and Arc Indices of\\
$\theta$-Curves and Handcuff Graphs}

\author{Cho, Eunchan$\mbox{}^*$ \hfill Shin, Jeongwon$\mbox{}^*$ \hfill Seo, Boyeon$\mbox{}^*$  \hfill Choi, Minho$\mbox{}^*$ \hfill Kim, Hun$\mbox{}^*$$\mbox{}^\ddagger$ \hfill Jin, GyoTaek$\mbox{}^\dagger$$\mbox{}^\ddagger$}

\institute{$\mbox{}^*$Korea Science Academy of KAIST \hfill $\mbox{}^\dagger$Department of Mathematical Science, KAIST \hfill $\mbox{}^\ddagger$Supervisor}

\logoleft{\includegraphics[height=10cm]{figure/logo.png}}

% ====================
% Footer (optional)
% ====================

\footercontent{
  \begin{center}
    \begin{minipage}[c]{\linewidth}
      \centering
      \makebox[0.3\linewidth][c]{%
        \includegraphics[height=2cm]{logos/gwagibu_logo.png}
      }%
      \makebox[0.4\linewidth][c]{%
        \includegraphics[height=2.1cm]{logos/image.png}
      }%
      \makebox[0.3\linewidth][c]{%
        \includegraphics[height=1.3cm]{logos/kaist_logo.png}
      }
    \end{minipage}
  \end{center}
}



% ====================
% Logo (optional)
% ====================

% use this to include logos on the left and/or right side of the header:
% \logoright{\includegraphics[height=2.4cm]{logos/utfpr-logo.png}}
% \logoleft{\hspace{20ex}\includegraphics[height=3.5cm]{logos/ppgca-logo.png}}

% ====================
% Body
% ====================



\begin{document}

% Refer to https://github.com/k4rtik/uchicago-poster
% logo: https://www.cam.ac.uk/brand-resources/about-the-logo/logo-downloads
% \addtobeamertemplate{headline}{}
% {
%     \begin{tikzpicture}[remember picture,overlay]
%       \node [anchor=north west, inner sep=3cm] at ([xshift=-2.4cm,yshift=1.75cm]current page.north west)
%       {\includegraphics[height=7cm]{logos/unott-logo.eps}}; 
%     \end{tikzpicture}
% }

% \begin{columns}[t]
% \separatorcolumn

\begin{frame}[t]

  % \v3space{2cm}
 \begin{alertblock}{Grid Diagrams of $\theta$-Curves and Handcuff Graphs}
  \begin{columns}[t]
  \begin{column}{0.6\textwidth}
  \begin{alertblock}{Theta-Curves}
  \begin{figure}
    \begin{subfigure}{0.05\textwidth}
    \includegraphics[width=2cm]{../Final_Poster/grid_diagram/theta_3_1.png}
    \caption{$3_1$} 
    \end{subfigure}
    \begin{subfigure}{0.05\textwidth}
    \includegraphics[width=1.5cm]{../Final_Poster/grid_diagram/theta_4_1.png}
    \caption{$4_1$} 
    \end{subfigure}
    \begin{subfigure}{0.05\textwidth}
    \includegraphics[width=1.5cm]{../Final_Poster/grid_diagram/theta_5_1.png}
    \caption{$5_1$}
    \end{subfigure}
    \begin{subfigure}{0.05\textwidth}
    \includegraphics[width=1.5cm]{../Final_Poster/grid_diagram/theta_5_2.png}
    \caption{$5_2$} 
    \end{subfigure}
    \begin{subfigure}{0.05\textwidth}
    \includegraphics[width=1.5cm]{../Final_Poster/grid_diagram/theta_5_3.png}
    \caption{$5_3$}
    \end{subfigure}
    \begin{subfigure}{0.05\textwidth}
    \includegraphics[width=1.5cm]{../Final_Poster/grid_diagram/theta_5_4.png}
    \caption{$5_4$} 
    \end{subfigure}
    \begin{subfigure}{0.05\textwidth}
    \includegraphics[width=1.5cm]{../Final_Poster/grid_diagram/theta_5_5.png}
    \caption{$5_5$} 
    \end{subfigure}
    \begin{subfigure}{0.05\textwidth}
    \includegraphics[width=1.5cm]{../Final_Poster/grid_diagram/theta_5_6.png}
    \caption{$5_6$} 
    \end{subfigure}
    \begin{subfigure}{0.05\textwidth}
    \includegraphics[width=1.5cm]{../Final_Poster/grid_diagram/theta_5_7.png}
    \caption{$5_7$} 
    \end{subfigure}
    \begin{subfigure}{0.05\textwidth}
    \includegraphics[width=1.5cm]{../Final_Poster/grid_diagram/theta_6_1.png}
    \caption{$6_1$} 
    \end{subfigure}
    \begin{subfigure}{0.05\textwidth}
    \includegraphics[width=1.5cm]{../Final_Poster/grid_diagram/theta_6_2.png}
    \caption{$6_2$} 
    \end{subfigure}
    \begin{subfigure}{0.05\textwidth}
    \includegraphics[width=1.5cm]{../Final_Poster/grid_diagram/theta_6_3.png}
    \caption{$6_3$} 
    \end{subfigure}
    \begin{subfigure}{0.05\textwidth}
    \includegraphics[width=1.5cm]{../Final_Poster/grid_diagram/theta_6_4.png}
    \caption{$6_4$} 
    \end{subfigure}
    \begin{subfigure}{0.05\textwidth}
    \includegraphics[width=1.5cm]{../Final_Poster/grid_diagram/theta_6_5.png}
    \caption{$6_5$} 
    \end{subfigure}
    \begin{subfigure}{0.05\textwidth}
    \includegraphics[width=1.5cm]{../Final_Poster/grid_diagram/theta_6_6.png}
    \caption{$6_6$} 
    \end{subfigure}
    \begin{subfigure}{0.05\textwidth}
    \includegraphics[width=1.5cm]{../Final_Poster/grid_diagram/theta_6_7.png}
    \caption{$6_7$} 
    \end{subfigure}
      \begin{subfigure}{0.05\textwidth}
    \includegraphics[width=1.5cm]{../Final_Poster/grid_diagram/theta_6_8.png}
    \caption{$6_8$} 
    \end{subfigure}
    \begin{subfigure}{0.05\textwidth}
    \includegraphics[width=1.5cm]{../Final_Poster/grid_diagram/theta_6_9.png}
    \caption{$6_9$} 
    \end{subfigure}
    \begin{subfigure}{0.05\textwidth}
    \includegraphics[width=1.5cm]{../Final_Poster/grid_diagram/theta_6_10.png}
    \caption{$6_{10}$} 
    \end{subfigure}
    \begin{subfigure}{0.05\textwidth}
    \includegraphics[width=1.5cm]{../Final_Poster/grid_diagram/theta_6_11.png}
    \caption{$6_{11}$} 
    \end{subfigure}
    \begin{subfigure}{0.05\textwidth}
    \includegraphics[width=1.5cm]{../Final_Poster/grid_diagram/theta_6_12.png}
    \caption{$6_{12}$} 
    \end{subfigure}
    \begin{subfigure}{0.05\textwidth}
    \includegraphics[width=1.5cm]{../Final_Poster/grid_diagram/theta_6_13.png}
    \caption{$6_{13}$} 
    \end{subfigure}
    \begin{subfigure}{0.05\textwidth}
    \includegraphics[width=1.5cm]{../Final_Poster/grid_diagram/theta_6_14.png}
    \caption{$6_{14}$} 
    \end{subfigure}
    \begin{subfigure}{0.05\textwidth}
    \includegraphics[width=1.5cm]{../Final_Poster/grid_diagram/theta_6_15.png}
    \caption{$6_{15}$} 
    \end{subfigure}
    \begin{subfigure}{0.05\textwidth}
    \includegraphics[width=1.5cm]{../Final_Poster/grid_diagram/theta_6_16.png}
    \caption{$6_{16}$} 
    \end{subfigure}
    \begin{subfigure}{0.05\textwidth}
    \includegraphics[width=1.5cm]{../Final_Poster/grid_diagram/theta_7_1.png}
    \caption{$7_{1}$} 
    \end{subfigure}
    \begin{subfigure}{0.05\textwidth}
    \includegraphics[width=1.5cm]{../Final_Poster/grid_diagram/theta_7_2.png}
    \caption{$7_{2}$} 
    \end{subfigure}
    \begin{subfigure}{0.05\textwidth}
    \includegraphics[width=1.5cm]{../Final_Poster/grid_diagram/theta_7_3.png}
    \caption{$7_{3}$} 
    \end{subfigure}
    \begin{subfigure}{0.05\textwidth}
    \includegraphics[width=1.5cm]{../Final_Poster/grid_diagram/theta_7_4.png}
    \caption{$7_{4}$} 
    \end{subfigure}
    \begin{subfigure}{0.05\textwidth}
    \includegraphics[width=1.5cm]{../Final_Poster/grid_diagram/theta_7_5.png}
    \caption{$7_{5}$} 
    \end{subfigure}
    \begin{subfigure}{0.05\textwidth}
    \includegraphics[width=1.5cm]{../Final_Poster/grid_diagram/theta_7_6.png}
    \caption{$7_{6}$} 
    \end{subfigure}
    \begin{subfigure}{0.05\textwidth}
    \includegraphics[width=1.5cm]{../Final_Poster/grid_diagram/theta_7_7.png}
    \caption{$7_{7}$} 
    \end{subfigure}
    \begin{subfigure}{0.05\textwidth}
    \includegraphics[width=1.5cm]{../Final_Poster/grid_diagram/theta_7_8.png}
    \caption{$7_{8}$} 
    \end{subfigure}
    \begin{subfigure}{0.05\textwidth}
    \includegraphics[width=1.5cm]{../Final_Poster/grid_diagram/theta_7_9.png}
    \caption{$7_{9}$} 
    \end{subfigure}
    \begin{subfigure}{0.05\textwidth}
    \includegraphics[width=1.5cm]{../Final_Poster/grid_diagram/theta_7_10.png}
    \caption{$7_{10}$} 
    \end{subfigure}
    \begin{subfigure}{0.05\textwidth}
    \includegraphics[width=1.5cm]{../Final_Poster/grid_diagram/theta_7_11.png}
    \caption{$7_{11}$} 
    \end{subfigure}
    \begin{subfigure}{0.05\textwidth}
    \includegraphics[width=1.5cm]{../Final_Poster/grid_diagram/theta_7_12.png}
    \caption{$7_{12}$} 
    \end{subfigure}
    \begin{subfigure}{0.05\textwidth}
    \includegraphics[width=1.5cm]{../Final_Poster/grid_diagram/theta_7_13.png}
    \caption{$7_{13}$} 
    \end{subfigure}
    \begin{subfigure}{0.05\textwidth}
    \includegraphics[width=1.5cm]{../Final_Poster/grid_diagram/theta_7_14.png}
    \caption{$7_{14}$} 
    \end{subfigure}
    \begin{subfigure}{0.05\textwidth}
    \includegraphics[width=1.5cm]{../Final_Poster/grid_diagram/theta_7_15.png}
    \caption{$7_{15}$} 
    \end{subfigure}
    \begin{subfigure}{0.05\textwidth}
    \includegraphics[width=1.5cm]{../Final_Poster/grid_diagram/theta_7_16.png}
    \caption{$7_{16}$} 
    \end{subfigure}
    \begin{subfigure}{0.05\textwidth}
    \includegraphics[width=1.5cm]{../Final_Poster/grid_diagram/theta_7_17.png}
    \caption{$7_{17}$} 
    \end{subfigure}
    \begin{subfigure}{0.05\textwidth}
    \includegraphics[width=1.5cm]{../Final_Poster/grid_diagram/theta_7_18.png}
    \caption{$7_{18}$} 
    \end{subfigure}
    \begin{subfigure}{0.05\textwidth}
    \includegraphics[width=1.5cm]{../Final_Poster/grid_diagram/theta_7_19.png}
    \caption{$7_{19}$} 
    \end{subfigure}
    \begin{subfigure}{0.05\textwidth}
    \includegraphics[width=1.5cm]{../Final_Poster/grid_diagram/theta_7_20.png}
    \caption{$7_{20}$} 
    \end{subfigure}
    \begin{subfigure}{0.05\textwidth}
    \includegraphics[width=1.5cm]{../Final_Poster/grid_diagram/theta_7_21.png}
    \caption{$7_{21}$} 
    \end{subfigure}
    \begin{subfigure}{0.05\textwidth}
    \includegraphics[width=1.5cm]{../Final_Poster/grid_diagram/theta_7_22.png}
    \caption{$7_{22}$} 
    \end{subfigure}
    \begin{subfigure}{0.05\textwidth}
    \includegraphics[width=1.5cm]{../Final_Poster/grid_diagram/theta_7_23.png}
    \caption{$7_{23}$} 
    \end{subfigure}
    \begin{subfigure}{0.05\textwidth}
    \includegraphics[width=1.5cm]{../Final_Poster/grid_diagram/theta_7_24.png}
    \caption{$7_{24}$} 
    \end{subfigure}
    \begin{subfigure}{0.05\textwidth}
    \includegraphics[width=1.5cm]{../Final_Poster/grid_diagram/theta_7_25.png}
    \caption{$7_{25}$} 
    \end{subfigure}
    \begin{subfigure}{0.05\textwidth}
    \includegraphics[width=1.5cm]{../Final_Poster/grid_diagram/theta_7_26.png}
    \caption{$7_{26}$} 
    \end{subfigure}
    \begin{subfigure}{0.05\textwidth}
    \includegraphics[width=1.5cm]{../Final_Poster/grid_diagram/theta_7_27.png}
    \caption{$7_{27}$} 
    \end{subfigure}
    \begin{subfigure}{0.05\textwidth}
    \includegraphics[width=1.5cm]{../Final_Poster/grid_diagram/theta_7_28.png}
    \caption{$7_{28}$} 
    \end{subfigure}
    \begin{subfigure}{0.05\textwidth}
    \includegraphics[width=1.5cm]{../Final_Poster/grid_diagram/theta_7_29.png}
    \caption{$7_{29}$} 
    \end{subfigure}
    \begin{subfigure}{0.05\textwidth}
    \includegraphics[width=1.5cm]{../Final_Poster/grid_diagram/theta_7_30.png}
    \caption{$7_{30}$}
    \end{subfigure}
    \begin{subfigure}{0.05\textwidth}
    \includegraphics[width=1.5cm]{../Final_Poster/grid_diagram/theta_7_31.png}
    \caption{$7_{31}$} 
    \end{subfigure}
    \begin{subfigure}{0.05\textwidth}
    \includegraphics[width=1.5cm]{../Final_Poster/grid_diagram/theta_7_32.png}
    \caption{$7_{32}$} 
    \end{subfigure}
    \begin{subfigure}{0.05\textwidth}
    \includegraphics[width=1.5cm]{../Final_Poster/grid_diagram/theta_7_33.png}
    \caption{$7_{33}$} 
    \end{subfigure}
    \begin{subfigure}{0.05\textwidth}
    \includegraphics[width=1.5cm]{../Final_Poster/grid_diagram/theta_7_34.png}
    \caption{$7_{34}$} 
    \end{subfigure}
    \begin{subfigure}{0.05\textwidth}
    \includegraphics[width=1.5cm]{../Final_Poster/grid_diagram/theta_7_35.png}
    \caption{$7_{35}$} 
    \end{subfigure}
    \begin{subfigure}{0.05\textwidth}
    \includegraphics[width=1.5cm]{../Final_Poster/grid_diagram/theta_7_36.png}
    \caption{$7_{36}$} 
    \end{subfigure}
    \begin{subfigure}{0.05\textwidth}
    \includegraphics[width=1.5cm]{../Final_Poster/grid_diagram/theta_7_37.png}
    \caption{$7_{37}$} 
    \end{subfigure}
    \begin{subfigure}{0.05\textwidth}
    \includegraphics[width=1.5cm]{../Final_Poster/grid_diagram/theta_7_38.png}
    \caption{$7_{38}$} 
    \end{subfigure}
    \begin{subfigure}{0.05\textwidth}
    \includegraphics[width=1.5cm]{../Final_Poster/grid_diagram/theta_7_39.png}
    \caption{$7_{39}$} 
    \end{subfigure}
    \begin{subfigure}{0.05\textwidth}
    \includegraphics[width=1.5cm]{../Final_Poster/grid_diagram/theta_7_40.png}
    \caption{$7_{40}$} 
    \end{subfigure}
    \begin{subfigure}{0.05\textwidth}
    \includegraphics[width=1.5cm]{../Final_Poster/grid_diagram/theta_7_41.png}
    \caption{$7_{41}$} 
    \end{subfigure}
    \begin{subfigure}{0.05\textwidth}
    \includegraphics[width=1.5cm]{../Final_Poster/grid_diagram/theta_7_42.png}
    \caption{$7_{42}$} 
    \end{subfigure}
    \begin{subfigure}{0.05\textwidth}
    \includegraphics[width=1.5cm]{../Final_Poster/grid_diagram/theta_7_43.png}
    \caption{$7_{43}$} 
    \end{subfigure}
    \begin{subfigure}{0.05\textwidth}
    \includegraphics[width=1.5cm]{../Final_Poster/grid_diagram/theta_7_44.png}
    \caption{$7_{44}$} 
    \end{subfigure}
    \begin{subfigure}{0.05\textwidth}
    \includegraphics[width=1.5cm]{../Final_Poster/grid_diagram/theta_7_45.png}
    \caption{$7_{45}$} 
    \end{subfigure}
    \begin{subfigure}{0.05\textwidth}
    \includegraphics[width=1.5cm]{../Final_Poster/grid_diagram/theta_7_46.png}
    \caption{$7_{46}$} 
    \end{subfigure}
    \begin{subfigure}{0.05\textwidth}
    \includegraphics[width=1.5cm]{../Final_Poster/grid_diagram/theta_7_47.png}
    \caption{$7_{47}$} 
    \end{subfigure}
    \begin{subfigure}{0.05\textwidth}
    \includegraphics[width=1.5cm]{../Final_Poster/grid_diagram/theta_7_48.png}
    \caption{$7_{48}$} 
    \end{subfigure}
    \begin{subfigure}{0.05\textwidth}
    \includegraphics[width=1.5cm]{../Final_Poster/grid_diagram/theta_7_49.png}
    \caption{$7_{49}$} 
    \end{subfigure}
    \begin{subfigure}{0.05\textwidth}
    \includegraphics[width=1.5cm]{../Final_Poster/grid_diagram/theta_7_50.png}
    \caption{$7_{50}$} 
    \end{subfigure}
    \begin{subfigure}{0.05\textwidth}
    \includegraphics[width=1.5cm]{../Final_Poster/grid_diagram/theta_7_51.png}
    \caption{$7_{51}$} 
    \end{subfigure}
    \begin{subfigure}{0.05\textwidth}
    \includegraphics[width=1.5cm]{../Final_Poster/grid_diagram/theta_7_52.png}
    \caption{$7_{52}$} 
    \end{subfigure}
    \begin{subfigure}{0.05\textwidth}
    \includegraphics[width=1.5cm]{../Final_Poster/grid_diagram/theta_7_53.png}
    \caption{$7_{53}$} 
    \end{subfigure}
    \begin{subfigure}{0.05\textwidth}
    \includegraphics[width=1.5cm]{../Final_Poster/grid_diagram/theta_7_54.png}
    \caption{$7_{54}$} 
    \end{subfigure}
    \begin{subfigure}{0.05\textwidth}
    \includegraphics[width=1.5cm]{../Final_Poster/grid_diagram/theta_7_55.png}
    \caption{$7_{55}$} 
    \end{subfigure}
    \begin{subfigure}{0.05\textwidth}
    \includegraphics[width=1.5cm]{../Final_Poster/grid_diagram/theta_7_56.png}
    \caption{$7_{56}$} 
    \end{subfigure}
    \begin{subfigure}{0.05\textwidth}
    \includegraphics[width=1.5cm]{../Final_Poster/grid_diagram/theta_7_57.png}
    \caption{$7_{57}$} 
    \end{subfigure}
    \begin{subfigure}{0.05\textwidth}
    \includegraphics[width=1.5cm]{../Final_Poster/grid_diagram/theta_7_58.png}
    \caption{$7_{58}$} 
    \end{subfigure}
    \begin{subfigure}{0.05\textwidth}
    \includegraphics[width=1.5cm]{../Final_Poster/grid_diagram/theta_7_59.png}
    \caption{$7_{59}$} 
    \end{subfigure}
    \begin{subfigure}{0.05\textwidth}
    \includegraphics[width=1.5cm]{../Final_Poster/grid_diagram/theta_7_60.png}
    \caption{$7_{60}$} 
    \end{subfigure}
    \begin{subfigure}{0.05\textwidth}
    \includegraphics[width=1.5cm]{../Final_Poster/grid_diagram/theta_7_61.png}
    \caption{$7_{61}$} 
    \end{subfigure}
    \begin{subfigure}{0.05\textwidth}
    \includegraphics[width=1.5cm]{../Final_Poster/grid_diagram/theta_7_62.png}
    \caption{$7_{62}$} 
    \end{subfigure}
    \begin{subfigure}{0.05\textwidth}
    \includegraphics[width=1.5cm]{../Final_Poster/grid_diagram/theta_7_63.png}
    \caption{$7_{63}$} 
    \end{subfigure}
    \begin{subfigure}{0.05\textwidth}
    \includegraphics[width=1.5cm]{../Final_Poster/grid_diagram/theta_7_64.png}
    \caption{$7_{64}$} 
    \end{subfigure}
    \begin{subfigure}{0.05\textwidth}
    \includegraphics[width=1.5cm]{../Final_Poster/grid_diagram/theta_7_65.png}
    \caption{$7_{65}$} 
    \end{subfigure}
  \end{figure}


  \end{alertblock}
\end{column}
\begin{column}{0.3\textwidth}
  \begin{alertblock}{Handcuff Graphs}
  \begin{figure}
    \begin{subfigure}{0.1\textwidth}
    \includegraphics[width=1.5cm]{../Final_Poster/grid_diagram/handcuff_2_1.png}
    \caption{$2_{1}$} 
    \end{subfigure}
    \begin{subfigure}{0.1\textwidth}
    \includegraphics[width=1.5cm]{../Final_Poster/grid_diagram/handcuff_4_1.png}
    \caption{$4_{1}$} 
    \end{subfigure}
    \begin{subfigure}{0.1\textwidth}
    \includegraphics[width=1.5cm]{../Final_Poster/grid_diagram/handcuff_5_1.png}
    \caption{$5_{1}$} 
    \end{subfigure}
    \begin{subfigure}{0.1\textwidth}
    \includegraphics[width=1.5cm]{../Final_Poster/grid_diagram/handcuff_6_1.png}
    \caption{$6_{1}$} 
    \end{subfigure}
    \begin{subfigure}{0.1\textwidth}
    \includegraphics[width=1.5cm]{../Final_Poster/grid_diagram/handcuff_6_2.png}
    \caption{$6_{2}$} 
    \end{subfigure}
    \begin{subfigure}{0.1\textwidth}
    \includegraphics[width=1.5cm]{../Final_Poster/grid_diagram/handcuff_6_3.png}
    \caption{$6_{3}$} 
    \end{subfigure}
    \begin{subfigure}{0.1\textwidth}
    \includegraphics[width=1.5cm]{../Final_Poster/grid_diagram/handcuff_6_4.png}
    \caption{$6_{4}$} 
    \end{subfigure}
    \begin{subfigure}{0.1\textwidth}
    \includegraphics[width=1.5cm]{../Final_Poster/grid_diagram/handcuff_6_5.png}
    \caption{$6_{5}$} 
    \end{subfigure}
    \begin{subfigure}{0.1\textwidth}
    \includegraphics[width=1.5cm]{../Final_Poster/grid_diagram/handcuff_6_6.png}
    \caption{$6_{6}$} 
    \end{subfigure}
    \begin{subfigure}{0.1\textwidth}
    \includegraphics[width=1.5cm]{../Final_Poster/grid_diagram/handcuff_6_7.png}
    \caption{$6_{7}$} 
    \end{subfigure}
    \begin{subfigure}{0.1\textwidth}
    \includegraphics[width=1.5cm]{../Final_Poster/grid_diagram/handcuff_6_8.png}
    \caption{$6_{8}$} 
    \end{subfigure}
    \begin{subfigure}{0.1\textwidth}
    \includegraphics[width=1.5cm]{../Final_Poster/grid_diagram/handcuff_6_9.png}
    \caption{$6_{9}$} 
    \end{subfigure}
    \begin{subfigure}{0.1\textwidth}
    \includegraphics[width=1.5cm]{../Final_Poster/grid_diagram/handcuff_7_1.png}
    \caption{$7_{1}$} 
    \end{subfigure}
    \begin{subfigure}{0.1\textwidth}
    \includegraphics[width=1.5cm]{../Final_Poster/grid_diagram/handcuff_7_2.png}
    \caption{$7_{2}$} 
    \end{subfigure}
    \begin{subfigure}{0.1\textwidth}
    \includegraphics[width=1.5cm]{../Final_Poster/grid_diagram/handcuff_7_3.png}
    \caption{$7_{3}$} 
    \end{subfigure}
    \begin{subfigure}{0.1\textwidth}
    \includegraphics[width=1.5cm]{../Final_Poster/grid_diagram/handcuff_7_4.png}
    \caption{$7_{4}$} 
    \end{subfigure}
    \begin{subfigure}{0.1\textwidth}
    \includegraphics[width=1.5cm]{../Final_Poster/grid_diagram/handcuff_7_5.png}
    \caption{$7_{5}$} 
    \end{subfigure}
    \begin{subfigure}{0.1\textwidth}
    \includegraphics[width=1.5cm]{../Final_Poster/grid_diagram/handcuff_7_6.png}
    \caption{$7_{6}$} 
    \end{subfigure}
    \begin{subfigure}{0.1\textwidth}
    \includegraphics[width=1.5cm]{../Final_Poster/grid_diagram/handcuff_7_7.png}
    \caption{$7_{7}$} 
    \end{subfigure}
    \begin{subfigure}{0.1\textwidth}
    \includegraphics[width=1.5cm]{../Final_Poster/grid_diagram/handcuff_7_8.png}
    \caption{$7_{8}$} 
    \end{subfigure}
    \begin{subfigure}{0.1\textwidth}
    \includegraphics[width=1.5cm]{../Final_Poster/grid_diagram/handcuff_7_9.png}
    \caption{$7_{9}$} 
    \end{subfigure}
    \begin{subfigure}{0.1\textwidth}
    \includegraphics[width=1.5cm]{../Final_Poster/grid_diagram/handcuff_7_10.png}
    \caption{$7_{10}$} 
    \end{subfigure}
    \begin{subfigure}{0.1\textwidth}
    \includegraphics[width=1.5cm]{../Final_Poster/grid_diagram/handcuff_7_11.png}
    \caption{$7_{11}$} 
    \end{subfigure}
    \begin{subfigure}{0.1\textwidth}
    \includegraphics[width=1.5cm]{../Final_Poster/grid_diagram/handcuff_7_12.png}
    \caption{$7_{12}$} 
    \end{subfigure}
    \begin{subfigure}{0.1\textwidth}
    \includegraphics[width=1.5cm]{../Final_Poster/grid_diagram/handcuff_7_13.png}
    \caption{$7_{13}$} 
    \end{subfigure}
    \begin{subfigure}{0.1\textwidth}
    \includegraphics[width=1.5cm]{../Final_Poster/grid_diagram/handcuff_7_14.png}
    \caption{$7_{14}$} 
    \end{subfigure}
    \begin{subfigure}{0.1\textwidth}
    \includegraphics[width=1.5cm]{../Final_Poster/grid_diagram/handcuff_7_15.png}
    \caption{$7_{15}$} 
    \end{subfigure}
    \begin{subfigure}{0.1\textwidth}
    \includegraphics[width=1.5cm]{../Final_Poster/grid_diagram/handcuff_7_16.png}
    \caption{$7_{16}$} 
    \end{subfigure}
    \begin{subfigure}{0.1\textwidth}
    \includegraphics[width=1.5cm]{../Final_Poster/grid_diagram/handcuff_7_17.png}
    \caption{$7_{17}$} 
    \end{subfigure}
    \begin{subfigure}{0.1\textwidth}
    \includegraphics[width=1.5cm]{../Final_Poster/grid_diagram/handcuff_7_18.png}
    \caption{$7_{18}$} 
    \end{subfigure}
    \begin{subfigure}{0.1\textwidth}
    \includegraphics[width=1.5cm]{../Final_Poster/grid_diagram/handcuff_7_19.png}
    \caption{$7_{19}$} 
    \end{subfigure}
    \begin{subfigure}{0.1\textwidth}
    \includegraphics[width=1.5cm]{../Final_Poster/grid_diagram/handcuff_7_20.png}
    \caption{$7_{20}$} 
    \end{subfigure}
    \begin{subfigure}{0.1\textwidth}
    \includegraphics[width=1.5cm]{../Final_Poster/grid_diagram/handcuff_7_21.png}
    \caption{$7_{21}$} 
    \end{subfigure}
    \begin{subfigure}{0.1\textwidth}
    \includegraphics[width=1.5cm]{../Final_Poster/grid_diagram/handcuff_7_22.png}
    \caption{$7_{22}$} 
    \end{subfigure}
    \begin{subfigure}{0.1\textwidth}
    \includegraphics[width=1.5cm]{../Final_Poster/grid_diagram/handcuff_7_23.png}
    \caption{$7_{23}$} 
    \end{subfigure}
    \begin{subfigure}{0.1\textwidth}
    \includegraphics[width=1.5cm]{../Final_Poster/grid_diagram/handcuff_7_24.png}
    \caption{$7_{24}$} 
    \end{subfigure}
    \begin{subfigure}{0.1\textwidth}
    \includegraphics[width=1.5cm]{../Final_Poster/grid_diagram/handcuff_7_25.png}
    \caption{$7_{25}$} 
    \end{subfigure}
    \begin{subfigure}{0.1\textwidth}
    \includegraphics[width=1.5cm]{../Final_Poster/grid_diagram/handcuff_7_26.png}
    \caption{$7_{26}$} 
    \end{subfigure}
    \begin{subfigure}{0.1\textwidth}
    \includegraphics[width=1.5cm]{../Final_Poster/grid_diagram/handcuff_7_27.png}
    \caption{$7_{27}$} 
    \end{subfigure}
    \begin{subfigure}{0.1\textwidth}
    \includegraphics[width=1.5cm]{../Final_Poster/grid_diagram/handcuff_7_28.png}
    \caption{$7_{28}$} 
    \end{subfigure}
    \begin{subfigure}{0.1\textwidth}
    \includegraphics[width=1.5cm]{../Final_Poster/grid_diagram/handcuff_7_29.png}
    \caption{$7_{29}$} 
    \end{subfigure}
    \begin{subfigure}{0.1\textwidth}
    \includegraphics[width=1.5cm]{../Final_Poster/grid_diagram/handcuff_7_30.png}
    \caption{$7_{30}$} 
    \end{subfigure}
    \begin{subfigure}{0.1\textwidth}
    \includegraphics[width=1.5cm]{../Final_Poster/grid_diagram/handcuff_7_31.png}
    \caption{$7_{31}$} 
    \end{subfigure}
    \begin{subfigure}{0.1\textwidth}
    \includegraphics[width=1.5cm]{../Final_Poster/grid_diagram/handcuff_7_32.png}
    \caption{$7_{32}$} 
    \end{subfigure}
    \begin{subfigure}{0.1\textwidth}
    \includegraphics[width=1.5cm]{../Final_Poster/grid_diagram/handcuff_7_33.png}
    \caption{$7_{33}$} 
    \end{subfigure}
    \begin{subfigure}{0.1\textwidth}
    \includegraphics[width=1.5cm]{../Final_Poster/grid_diagram/handcuff_7_34.png}
    \caption{$7_{34}$} 
    \end{subfigure}
    \begin{subfigure}{0.1\textwidth}
    \includegraphics[width=1.5cm]{../Final_Poster/grid_diagram/handcuff_7_35.png}
    \caption{$7_{35}$} 
    \end{subfigure}
    \begin{subfigure}{0.1\textwidth}
    \includegraphics[width=1.5cm]{../Final_Poster/grid_diagram/handcuff_7_36.png}
    \caption{$7_{36}$} 
    \end{subfigure}
  \end{figure}
  \end{alertblock}
  \end{column}
  \end{columns}
  \end{alertblock}

  \vspace*{-2.5cm}

% \begin{column}{\colwidth}
\begin{columns}[t]
  \column{0.33\textwidth}
  \begin{block}{Definitions}
    %\textbf{Knot} is an embedding of a circle in 3-dimensional Euclidean space.
    %In \textbf{Knot Theory}, people classify each knot using their \textbf{Knot Invariant}. \\
    \begin{itemize}
      \item \textbf{$\theta$-curve}
    \end{itemize}
      \begin{figure}[h]
      \centering
      \includegraphics[width=\textwidth]{figure/theta_exe.png}
      \end{figure}
    \begin{itemize}
      \item \textbf{Handcuff graph}
    \end{itemize}
    \vspace*{-1cm}
      \begin{figure}[h]
      \centering
      \includegraphics[width=\textwidth]{figure/handcufflist.png}
      \end{figure}
  \end{block}

\vspace*{1.4cm}

  \begin{block}{Diagrams}
    %\caption{An algorithm with caption}\label{alg:cap}
    \begin{itemize} 
      \item \textbf{Grid diagram and Arc presentation}
    \end{itemize}
    \vspace*{-1cm}
      \begin{figure}[h]
        \centering
        \begin{tabular}{cc}
          \includegraphics[width=0.6\textwidth]{figure/grid_diagram_1.png} &
          \includegraphics[width=0.2\textwidth]{figure/grid_diagram_2.png} \\
        \end{tabular}        
      \end{figure}
    \vspace*{-1cm}
    \begin{itemize}
      \item \textbf{Arc index} : The minimal number of pages among all possible arc presentations of K
    \end{itemize}
  \end{block}

\vspace*{1.4cm}

  \begin{block}{Matrices of $\theta$-Curves and Handcuff Graphs}
  \begin{itemize}
    \item \textbf{Cromwell matrix} :\\an $n\times n$ binary matrix each of whose rows and columns has exactly two 1s corresponding a grid diagram of a knot
    \item \textbf{THC-Cromwell matrix} :\\expansion of Cromwell matrix into $\theta$-curves and handcuff graphs
    \item \textbf{H-deletion matrix}
  \end{itemize}
  \vspace*{-1cm}
  \begin{figure}[h]
    \centering
    \includegraphics[width=0.8\textwidth]{figure/Hdeletion.png}
    \caption{H-deletion of $\theta$-curve graph $3_1$}
  \end{figure}


  \end{block}
  \column{0.33\textwidth}

  
  % \begin{block}{Stacked Tangle Representation}
  %   \begin{itemize}
  %     \item \textbf{Non-simple disk} : A disk which contains one vertex and three line segments which joins the vertex and boundary point. 
  %     \item \textbf{Simple disk} : A disk which contains simple arc which joins two points on the boundary.
  %     \item \textbf{Stacked tangle} : Stacked disks each with the frame as boundary. Only two disks of the stacked tangle are non-simple disks(one is at the top), and others are simple disks. \\
  %     %\item \textbf{Arcs} : Segments properly embedded in each disk, with endpoints on the boundary
  %     \item \textbf{Caps} : Simple arc in outside of stacked tangle joining end points of arcs or line segments. \\
  %   \end{itemize}
  %   \begin{figure}
  %     \centering
  %     \includegraphics[width=0.4\textwidth]{figure/stacked_theta.png}
  %     \caption{Stacked tangle of handcuff graph}
  %   \end{figure}
  %   \begin{itemize}
  %     \item An arc in simple disk and arc in other simple disk or tree in non-simple disk intersect at most one point, and any two caps have no intersection when viewed from above.
  %   \end{itemize}
  %   \begin{figure}
  %     \centering
  %     \includegraphics[width=0.7\textwidth]{figure/stacked_tangle.png}
  %     \caption{Stacked tangle and simple closure of handcuff graph $2_1$}
  %   \end{figure}
  %   \begin{itemize}
  %     \item \textbf{Reduced simple closure of a stacked tangle} : A simple closure of a stacked tangle without any nested caps. Any two arcs (including line segment) joining by caps have no intersection when viewed from above.
  %   %\item \textbf{Simple closure}: The planar projection of the stacked tangle without any nested caps
  %   \end{itemize}
  % \end{block}
  \begin{block}{Determinants}

  $\mathbf{Theorem\ 1.}$ Let $K$ be a knot. Then the determinant of a Cromwell matrix of $K$ is $0$ or $\pm 2$.
  \begin{figure}[h]
    \centering
    \begin{tabular}{cccc}
      \includegraphics[width=0.2\textwidth]{figure/trefoilimage.png} &
      \includegraphics[width=0.2\textwidth]{figure/gridtheta.png} &
      \includegraphics[width=0.24\textwidth]{figure/matrix2.png} &
      \includegraphics[width=0.24\textwidth]{figure/matrix3.png} \\
    \end{tabular}  
    \caption{Determinant of a Cromwell matrix of Trefoil}
  \end{figure}

  $\mathbf{Theorem\ 2.}$ Let $M$ be a THC-Cromwell matrix of $\theta$-curve or handcuff graph, and let det*($M$) be the determinant of H-deletion matrix of $M$. Then, \\
		\begin{itemize}
			\item det*($M$) = $\pm1$ $\iff$ $M$ represents $\theta$-curve
			\item det*($M$) = $0$ or $\pm2$ $\iff$ $M$ represents handcuff graph
		\end{itemize}
    \vspace*{-2cm}
  %\begin{figure}
   % \centering
    %\includegraphics[height=8.5cm]{figure/table.png}
  %\end{figure}
  \vspace*{0.5cm}
  \begin{figure}[h]
  \centering
	\begin{tabular}{cccc}
    \includegraphics[width=0.25\textwidth]{figure/Line-shape.png} &
    \includegraphics[width=0.25\textwidth]{figure/matrix4.png} &
    \includegraphics[width=0.21\textwidth]{figure/matrix5.png} &
    \includegraphics[width=0.23\textwidth]{figure/matrix6.png} \\
    \includegraphics[width=0.25\textwidth]{figure/T-shape.png} &
    \includegraphics[width=0.25\textwidth]{figure/matrix7.png} &
    \includegraphics[width=0.21\textwidth]{figure/matrix8.png} &
    \includegraphics[width=0.22\textwidth]{figure/matrix9.png} \\
  \end{tabular}
  \vspace*{-0.5cm}
  \caption{Determinant of a $\theta$-curve : Line-shape and T-shape}
  \end{figure}
  \vspace*{-2cm}
  \begin{figure}[h]
  \centering
	\begin{tabular}{cccc}
    \includegraphics[width=0.23\textwidth]{figure/T-loop.png} &
    \includegraphics[width=0.25\textwidth]{figure/matrix10.png} &
    \includegraphics[width=0.24\textwidth]{figure/matrix11.png} &
    \includegraphics[width=0.23\textwidth]{figure/matrix12.png} \\
  \end{tabular}
  \vspace*{-0.5cm}
  \caption{Determinant of a Handcuff graph}
  \vspace*{-1.5cm}
  \end{figure}
  \vspace*{0.5cm}
  \end{block}

\vspace*{1.4cm}

  \begin{block}{Upper Bounds of Arc Index}
  $\mathbf{Notation.}$
  \vspace{-0.7cm}
  For a spatial graph $G$,
  \begin{itemize}
    \item $c(G)$: the number of crossing of $G$
    \item $\alpha(G)$: arc index of $G$
  \end{itemize}

  $\mathbf{Theorem\ 3.}$ $[$3$]$ For a spatial graph $G$, 
  \begin{equation*}
    \alpha(G) \leq c(G) + e + b
  \end{equation*}
  where $e$ and $b$ are the number of edges and bouquet cut-components, repsectively.
  
  $\mathbf{Corollary\ 4.}$ For non-split handcuff graph $H$,
  \begin{equation*}
    \alpha(H) \leq c(H) + 3.
  \end{equation*}

 % $\mathbf{Theorem.}$ If H is handcuff, then 
 % \begin{equation*}
  %  \alpha(H) \geq \frac{1+\sqrt{f(\alpha(H)) + 36c(H)}}{3}
 % \end{equation*}
  %where
 % \begin{equation*}
  %  f(x) =
   % \begin{cases}
  %  73 & (x \equiv 0 \pmod{6}) \\
 %   4 & (x \equiv 1 \pmod{6}) \\
 %   25 & (x \equiv 2 \pmod{6}) \\
 %   -8 & (x \equiv 3 \pmod{6}) \\
 %   49 & (x \equiv 4 \pmod{6}) \\
  %  -20 & (x \equiv 5 \pmod{6})
 %   \end{cases}
 % \end{equation*}

  %$\mathbf{Theorem.}$ If $L$ is non-split alternating link, then 
  %\begin{equation*}
   % \alpha(L) = c(L)+2
  %\end{equation*}

  \end{block}

  \column{0.33\textwidth}
  \begin{block}{Lower Bounds of Arc Index}
    
    $\mathbf{Theorem\ 5.}$ Let $H$ be a handcuff graph and $L$ be a constituent link of $H$. If $L$ is non-split and alternating, then 
  \begin{equation*}
    \alpha(H) \geq c(L) + 3.
  \end{equation*}

    $\mathbf{Definition\ 6.}$ $[$4$]$ Let $D_G$ be a diagram of a $\theta$-curve of handcuff-graph $G$. The \textbf{Yamada poplynomial} $R(G) \in \mathbb{Z}[x^{\pm{1}}]$ is calculated by \\
    \vspace*{0.5cm}
    \includegraphics[width=\textwidth]{figure/yamada.png} \\
    \vspace*{0.5cm}
    \begin{center}
    \includegraphics[width=0.9\textwidth]{figure/stackeds.png}
    \end{center}

    $\mathbf{Theorem\ 7.}$ Let $S_G$ be a simple closure of stacked tangle of a $\theta$-curve or handcuff graph $G$. And let $c$ and $n$ be the number of caps and crossings in $S_T$, respectively. Then,
     \[ \mathrm{spr}(R(S_T)) \leq c+2n \]
     where $\mathrm{spr}(f)$ denotes the difference between the maximal and minimal degrees of $f$.

    $\mathbf{Theorem\ 8.}$ Let $G$ be a $\theta$-curve or handcuff graph. Then the arc index of G is given by
     \[ \alpha(G) \ge 2 + \sqrt{\mathrm{spr}(R(S_G)) - 4}. \]
    \vspace*{-1.5cm}
  \end{block}

  \begin{block}{Further Studies}
    \begin{itemize}
      \item Completely determine the arc indices.
      \item Applying the result to other objects.
    \end{itemize}
    \vspace*{-1cm}
  \end{block}

  \begin{block}{References}
{\small
    \begin{description}
      \item[1] Moriuchi, H. (2019). A table of $\theta$-curves and handcuff graphs with up to seven crossings. \textit{Advanced Studies in Pure Mathematics}, 281–290.
      \item[2] Yoonsang Lee. (2023). A Study on Arc Index of Theta-Curves. Korea Science Academy of KAIST
      \item[3] Minjung Lee, Sungjong No, and Seungsang Oh. (2018). Arc index of spatial graphs. \textit{Journal of Graph Theory, 90}(3), 406–415.
      \item[4] Yamada, S. (1989). An invariant of spatial graphs. \textit{Journal of Graph Theory, 13}(5), 537–551.      
    \end{description}
}
	% \TINY
	% \bibliographystyle{apalike}
	% \nocite{*}
	% \bibliography{reference}

  \end{block}

\end{columns}


\end{frame}
\end{document}