\section{Lower Bounds of Arc Index}

\begin{frame}{Lower Bound from Constituent Knots/Links}
	\bigskip
	\begin{itemize}
      	\item A \term{constituent knot $T_{ij}$}, $1 \le i < j \le 3$, is a subgraph of $\theta$-curve $T$
      that consists of two vertices $v_1$, $v_2$ and two edges $e_i$, $e_j$.
		\begin{tabu}{X[c,8]X[c,8]X[c,8]X[c,8]}
			\includegraphics[width=0.8\linewidth]{theta.png}
			& \includegraphics[width=0.8\linewidth]{theta1.jpg}
			& \includegraphics[width=0.8\linewidth]{theta2.jpg}
			& \includegraphics[width=0.8\linewidth]{theta3.jpg}
		\end{tabu}

        \item A \term{constituent link $H_{12}$}, is a subgraph of handcuff graph $H$
      that consists of two vertices $v_1$, $v_2$ and two edges $e_1$, $e_2$.
		\begin{tabu}{X[c,8]X[c,8]}
			\includegraphics[width=0.5\linewidth]{hand.jpg}
			& \includegraphics[width=0.5\linewidth]{hand1.jpg}
		\end{tabu}

	\end{itemize}
\end{frame}

\begin{frame}{Lower Bound from Constituent Knots/Links}
	\begin{thm}[Lee^{\dagger}, 2023]
		Let $T$ be a $\theta$-curve
		and $K_1$, $K_2$, $K_3$ be three constituent knots of $T$.
		Then
		\[
			\alpha(T) \ge \max_{i\in\{1,2,3\}} \alpha(K_i) + 1
		\]
	\end{thm}
	
	\begin{thm}[Lee^{\dagger}, 2023]
		Let $T$ be a $\theta$-curve
		and $K_1$, $K_2$, $K_3$ be three constituent knots of $T$.
		Then
		\[
			\centerline{$\displaystyle\alpha(T) \ge \frac12 \sum_{i=1}^3\alpha(K_i)$}
		\]
	\end{thm}
\end{frame}

\begin{frame}{Lower Bound from Constituent Knots/Links}
	\bigskip
	\begin{thm}
		Let $H$ be a handcuff graph, and $L$ be the constituent link of $H$.
		If $L$ is an alternating and non-split link, then
		\[
			\alpha(H) \ge c(L) + 3,
		\]
		where $c(L)$ is the crossing number of $L$.
	\end{thm}

	%\begin{corollary}
	%	Let $H$ be a handcuff graph, and $L$ be the constituent link of $H$. If $L$ is alternating and non-split, then
	%	\[
	%		\alpha(H) = c(L) + 3.
	%	\]
	%\end{corollary}

	\center
	\includegraphics[width=0.3\linewidth]{hopf.png}
\end{frame}

\begin{frame}{Stacked Tangle of an $\theta$-Curve and Handcuff Graph}
	\begin{tabu}{X[c,8]X[c,8]X[c,10]X[c,10],X[c,10]}
		\includegraphics[width=\linewidth]{stacked_tangle.png}
		& \includegraphics[width=\linewidth]{stacked_tangle2.png}
		& \includegraphics[width=\linewidth]{stacked.png}
		& \includegraphics[width=\linewidth]{stacked_theta.png}
		& \includegraphics[width=.3\linewidth]{stacked_handcuff.png} \\
		\multicolumn{2}{c}{Stacked Tangle of a Link}
		& \multicolumn{2}{c}{$\theta$-Curve}
		& Handcuff Graph
	\end{tabu}

	%\seprule
\end{frame}



\begin{frame}{Yamada Polynomials}
	Let $D_G$ be a diagram of an $\theta$-curve or handcuff graph $G$.
	Then, the \term{Yamada Polynomial $R(D_G)\in \mathbf{Z}\left[x^{\pm1}\right]$} is calculated by the following properties:
	\begin{itemize}
		\item \myem{Y6:} $R\left(\raisebox{-3pt}{\includegraphics[height=12pt]{y6}}\right)= - (x+1+x^{-1})(x + x^{-1}) = -x^2 - x - 2 - x^{-1} - x^{-2}$\hfill \myem{Y7:} $R\left(\raisebox{-1.5pt}{\includegraphics[height=8pt]{y7}}\right)=0$
		\item \myem{Y8:} $R(G\cup\bigcirc) = (x+1+x^{-1})R(G)$ for an arbitrary $\theta$-curve or handcuff graph diagram $G$
		\item \myem{Y9:} $R\left(\raisebox{-3pt}{\includegraphics[height=12pt]{y91}}\right)-R\left(\raisebox{-3pt}{\includegraphics[height=12pt]{y92}}\right)=(x-x^{-1})\left[R\left(\raisebox{-3pt}{\includegraphics[height=12pt]{y93}}\right)-R\left(\raisebox{-3pt}{\includegraphics[height=12pt]{y94}}\right)\right]$
		\item \myem{Y10:} $R\left(\raisebox{-3pt}{\includegraphics[height=12pt]{y101}}\right) = x^2 R\left(\raisebox{-3pt}{\includegraphics[height=12pt]{y103}}\right)$,\quad
		$R\left(\raisebox{-3pt}{\includegraphics[height=12pt]{y102}}\right) = x^{-2} R\left(\raisebox{-3pt}{\includegraphics[height=12pt]{y103}}\right)$
		\item \myem{Y11:} $R\left(\raisebox{-3pt}{\includegraphics[height=12pt]{y111}}\right) = R\left(\raisebox{-3pt}{\includegraphics[height=12pt]{y93}}\right)$\hfill
		\myem{Y12:} $R\left(\raisebox{-3pt}{\includegraphics[height=12pt]{y121}}\right) = R\left(\raisebox{-3pt}{\includegraphics[height=12pt]{y122}}\right)$
		\item \myem{Y13:} $R\left(\raisebox{-3pt}{\includegraphics[height=12pt]{y131}}\right) = R\left(\raisebox{-3pt}{\includegraphics[height=12pt]{y132}}\right)$,\quad $R\left(\raisebox{-3pt}{\includegraphics[height=12pt]{y133}}\right) = R\left(\raisebox{-3pt}{\includegraphics[height=12pt]{y134}}\right)$
		\item \myem{Y14:} $R\left(\raisebox{-3pt}{\includegraphics[height=12pt]{y141}}\right) = -x R\left(\raisebox{-3pt}{\includegraphics[height=12pt]{y143}}\right)$,\quad $R\left(\raisebox{-3pt}{\includegraphics[height=12pt]{y142}}\right) = -x^{-1}R\left(\raisebox{-3pt}{\includegraphics[height=12pt]{y143}}\right)$
	\end{itemize}

	\begin{prop}[Yamada, 1989]
		$R(D_G)$ is an ambient isotopy invariant of $G$ up to multiplying $(-x)^n$ for some integer $n$.
	\end{prop}
\end{frame}


\begin{frame}{Lower Bound from Yamada Polynomial}
	\begin{thm}
	Let $G$ be a $\theta$-curve or handcuff graph.
	Then
	\[
		2 + \sqrt{| \max\deg_xR(S_G) - \min\deg_xR(S_G) - 4 |} \le \alpha(G).
	\]
	\end{thm}	
\end{frame}


\begin{frame}{Sketch of Proof}
	\begin{prop}
		Let $S_G$ be a simple closure of stacked tangle of a $\theta$-curve or handcuff graph $G$ \myem{without nested caps}.
		Then
		\[
			\max\deg_x R(S_G) \le c + n, \quad
			\min\deg_x R(S_G) \ge -(c + n),
		\]
		where \myem{$c$} and \myem{$n$} is the number of caps and crossings in $S_G$, respectively.
	\end{prop}

	\mypf
	\begin{itemize}
		\item Let \myem{$c_s$} and \myem{$c_{ss}$} be the number of \myem{simple caps} and \myem{semi-simple caps}, respectively.
		\item Use double mathematical induction of $(c_s + c_{ss}, n)$.
        $$\includegraphics[width=.35\linewidth]{simplecap.png}$$
	\end{itemize}
\end{frame}

\begin{frame}{Sketch of Proof}
	\begin{prop}
		Let $S_G$ be a reduced simple closure of stacked tangle of a $\theta$-curve or handcuff graph $G$
		corresponding to a minimal arc presentation of $G$.
		Then
		\[
			\max\deg_xR(S_G) - \min\deg_xR(S_G) - 2n \le \alpha(G)
			% \max\deg_xR(S_T) - \min\deg_xR(S_T) -2n + 4 \le 3\alpha(T)
		\]
		where $n$ is the number of crossings in $S_G$.
	\end{prop}
	\mypf

	\begin{itemize}
		\item $S_G$ is a reduced simple closure of stacked tangle corresponding to a minimal arc presentation.
		\item The number of caps $c$ in $S_G$ is exactly the arc index of $G$, $\alpha(G)$.
	\end{itemize}
\end{frame}


\begin{frame}{Sketch of Proof}
	\begin{itemize}
		\item Take a cap and add a positive or negative curl
		$$\raisebox{-1cm}{\includegraphics[height=2cm]{ST1_new}} \quad \longrightarrow \quad \raisebox{-1cm}{\includegraphics[height=2cm]{ST2_new}}$$
		% $$\raisebox{-1cm}{\includegraphics[height=2cm]{ST1}} \quad \longrightarrow \quad \raisebox{-1cm}{\includegraphics[height=2cm]{ST2}} \quad \longrightarrow \quad \raisebox{-1cm}{\includegraphics[height=2cm]{ST3}}$$

	\vspace{0.3cm}

	$$
	\begin{tabu} to .6\linewidth {X[2,l]|X[c]|X[c]} \hline
	 & $S_G^{neg}$ & $S_G^{pos}$\\\hline
	 Number of Caps & $c$ & $c$\\\hline
	 Number of Crossings & $n+p$ & $n+(c-p)$ \\\hline
	\end{tabu}
	$$

	\vspace{0.3cm}

	\begin{align*}
		\min\deg_xR(S_G) - 2p = \min\deg_xR\left(S_G^{neg}\right) &\ge -c + - (n+p)\\
		\max\deg_xR(S_G) + 2(c-p) = \max\deg_xR\left(S_G^{pos}\right) &\le c + \left[n+(c-p)\right]\\
		\max\deg_xR(S_G) - \min\deg_xR(S_G) & \le c + 2n \quad \quad \quad \quad \quad \quad \quad \qed
	\end{align*}

	\end{itemize}
\end{frame}


\begin{frame}{Sketch of Proof}
	\begin{thm}
	Let $G$ be a $\theta$-curve or handcuff graph.
	Then
	\[
		2 + \sqrt{| \max\deg_xR(S_G) - \min\deg_xR(S_G) - 4 |} \le \alpha(G).
	\]
	\end{thm}	

	\mypf

	Let $S_G$ be a reduced simple closure of stacked tangle of a $\theta$-curve or handcuff graph $G$
	corresponding to a minimal arc presentation of $G$.

	\begin{itemize}
		\item The number of caps : $\alpha(G)$
		\item The number of non-simple disks : $2$
		\item The number of simple disks : $\alpha(G)-3$
	\end{itemize}

	The theorem is followed by considering the maximum of crossings in $S_G$ and applying previous theorems.
\end{frame}